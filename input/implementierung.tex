\chapter{Implementierung}
\section{Implementierung der Schnittstelle}
\section{Umsetzung der Fahrphysik}
\begin{figure}[htp]
    \centering
    \includegraphics[width=(\textwidth/2)]{images/AGV_Vier_Sensoren.PNG}
    \caption{AGV Aufbau}
    \label{fig:Aufbau}
\end{figure}

%Warum wird nutzt die ThalesApi gRPC?
Die ThalesApi wird verwendet um die Sensordaten der Test Sceneie weiterzuleiten.
Die ThalesApi gRPC verwendet das Protocol Buffers (protobuf) Datenformat, das binär kodiert ist und daher effizienter als JSON oder XML ist. Dies führt zu schnellerer Datenübertragung und geringerem Ressourcenverbrauch, was in Spielen und anderen Unity-Anwendungen wichtig ist.\\
Das gRPC Protokocol welches die ThalesApi verwendet bietet Unterstützung für eine Vielzahl von Programmiersprachen, darunter C\#, was die Integration in Unity erleichtert. Dies ermöglicht es, Client-Server-Kommunikation nahtlos zwischen Unity und anderen Serveranwendungen in verschiedenen Sprachen herzustellen. Die Clients können in vielen Sprachen geschrieben werden. Im DSM Projekt wird ein C++ CLient verwendet.\\
Mit gRPC können Sie starke Typisierung für Ihre Dienstaufrufe verwenden. Dies bedeutet, dass Sie spezifische Methodenaufrufe mit stark typisierten Eingabe- und Ausgabeparametern definieren können, was die Entwicklung und Wartung Ihrer Anwendung erleichtert.\\
gRPC unterstützt bidirektionale Streaming-Kommunikation, was bedeutet, dass sowohl der Client als auch der Server Daten an den anderen senden können, ohne auf eine Anfrage des anderen zu warten. Dies ist besonders nützlich für Multiplayer-Spiele und Echtzeit-Anwendungen.\\
gRPC bietet Tools zur automatischen Codegenerierung, um Client- und Serverstubs sowie die erforderlichen Datenklassen basierend auf Ihren Protobuf-Definitionen zu erstellen. Dies vereinfacht die Entwicklung und verhindert Fehler durch manuelle Codeerstellung. Dies wird vorallem für die CustumApi genutzt, um beliebige Variablen zuübertragen.\\
%An meinem Tutorial orientieren
Die ThalesApi basiert auf gRPC. Die ThalesApi erstellt einen gRPC Server dessen Services von einem Client abgerufen werden.  

%wie funktioniert die ThalesApi?
Sobald die ThalesApi der Scene hinzugefügt wird können die Sensordaten mit dem Client abgerfagt werden.

%Warum muss ein Autostepper verwendet werden?
%Wie wird der Sensor in Konfiguriet

%Wann wird diese Aktiv?
\subsection{Übertragung von den Sensordaten}
%Wie wurde es nochmal genau gemacht?
Die Sensor daten werden über den Client abgerufen.

\begin{lstlisting}[language={[Sharp]C}, caption={Client}, label={Script}]
int main(int argc, char** argv) {
    auto consoleOutput = std::make_shared<sick::dsm::consumers::ConsoleOutputConsumer>();
    auto slaveProxy0 = std::make_shared<sick::dsm::proxies::SlaveProxy>("127.0.0.1", 5555, 6555);
    slaveProxy0->setSendImmediately(true).setMaxSlices(1).setSliceBeamCount(2200).setMtuSize(1460).setIndex(0).setIdentifier({127, 0, 0, 1, 0x15, 0xb3, 0, 0}).initialise();
    ...
    sick::dsm::proxies::MasterProxy masterProxy0(5555, {127, 0, 0, 1, 0x15, 0xb3, 0, 0});
    ...
    sick::dsm::proxies::MasterProxy masterProxy3(5558, {127, 0, 0, 1, 0x15, 0xb6, 0, 0});
    masterProxy3.addConsumer(recorder3);
    scanner0->start();
    ...
    int counter = 0;
    std::thread t1([&](){
        while (true) {
            scanner0->produce();
            ...
        }
    });
    std::thread t2([&](){
        while (true) {
            masterProxy0.produce();
            ...
        }
    });
    t1.join();
    t2.join();
    return 0;
}
\end{lstlisting}
Der Code beginnt mit Include-Deklarationen, die die notwendigen Header-Dateien für die Verwendung von C++-Standardbibliotheksfunktionen und anderen Bibliotheken wie gRPC und der "sick::dsm"-Bibliothek importieren.\\

Im weiteren Verlauf des Codes werden verschiedene Variablen und Objekte deklariert. Dazu gehören Objekte wie consoleOutput, slaveProxy0, slaveProxy1, slaveProxy2, slaveProxy3, channel, scanner0, scanner1, scanner2, scanner3, recorder0, recorder1, recorder2, recorder3, masterProxy0, masterProxy1, masterProxy2 und masterProxy3. Diese Objekte repräsentieren verschiedene Teile der Anwendung, darunter die Kommunikation mit Laserscannern, die Aufzeichnung von Daten und die Verwendung von gRPC für die Kommunikation.\\

Die Slave-Proxies (slaveProxy0, slaveProxy1, slaveProxy2, slaveProxy3) werden konfiguriert, um Verbindungen zu bestimmten IP-Adressen und Ports herzustellen. Diese Proxies werden wahrscheinlich verwendet, um Daten von den Laserscannern zu empfangen und zu verarbeiten.\\

Eine gRPC-Verbindung (channel) zu einer bestimmten IP-Adresse und Port wird erstellt. Dies wird wahrscheinlich für die Kommunikation mit einem entfernten Server verwendet.\\

Es werden Scanner-Objekte (scanner0, scanner1, scanner2, scanner3) erstellt und für die Verarbeitung von Daten aus den Laserscannern konfiguriert. Sie fügen auch Slave-Proxies als Verbraucher hinzu, um die verarbeiteten Daten weiterzuleiten.\\

Recorder-Objekte (recorder0, recorder1, recorder2, recorder3) werden erstellt. Diese Objekte zeichnen Daten in Dateien auf.\\

Master-Proxies (masterProxy0, masterProxy1, masterProxy2, masterProxy3) werden konfiguriert, um Daten von den Slave-Proxies zu empfangen.\\

Es werden zwei Threads (t1 und t2) erstellt, die kontinuierlich Daten von Scannern und Master-Proxies produzieren. Diese Threads laufen in einer Endlosschleife und führen die entsprechenden Produktionsschritte aus.\\
%Es wurde eine Subsriction erstellt
%Die Sensordaten werden abgefragt sondern sie sollen alle daten geschickt werden die aufgenommen wurden
%Wie wurde der Client gestalltet?
%Client mit ChatGpt kommentieren lassen
\begin{figure}[htp]
    \centering
    \includegraphics[width=(\textwidth/2)]{images/Mit_Beams.PNG}
    \caption{AGV mit sichtbaren Beams}
    \label{fig:Mit Beams}
\end{figure}



\subsection{Übertragung von Variablen}
%Welche Variablen?
Mit dem Oben gezeigten Client empfängt die Sensordaten über die Thales gRPC Server. Um den Lenkwinkel und die Geschwindigkeit zu übertragen ist eine CustumAPI notwendig.
%Was ist ein CustumAPI?
Wenn die ThalesAPI nicht die Anforderungen erfüllt gibt es die Möglichkeit sie zuerweitern. Die generierung dieser CustumAPI findet außerhalb dieser Projektarbeit statt. Die CustumAPI kann durch die Assts in das Unity Projekt eingebunden werden.
%Wie erstellt man eine CustumApi?
%Wie bindet man die Custumapi in Unity ein?
Es wird dem Objekt das CustumAPI-Script hinzugefügt und als öffentliche Variable dem Skript zugeordnet, damit es als verfügbarer Service auf dem Server erscheint.
%Man fügt das CustumAPi Srcipt dem Objekt zu fügt das Objekt als Public variable dem Script zu damit sollte es als verfübarer Service auf dem Server erscheinen

%Warum nutzt man die CustumApi?
%Wie wird die CustumApi in Unity verwendet?
%Wo für verwendete die CustumApi?


%\section{AGVs in der virtuellen Umgebung}
\section{Umsetzung der Fahrphysik}



\subsection{Lenkung}
%Wie funktioneriert die Lenkung mit Ackermann winkel
Der Lenk mechanismus wird durch den Ackermannwinkel umgesetzt.
%Warum wurde dieser verwendetet?
Der Ackermannwinkel wird verwendet weil das Äußere Rat und das innere Rad einen anderen Wendekreis. Wenn die beiden Räder den gleichen Einschlagwinkel haben würde das Rad mit der geringeren haftung zum boden den Kontakt zudiesem verlieren.\\
\begin{figure}[htp]
    \centering
    \includegraphics[width=(\textwidth/2)]{images/Ackermann.png}
    \caption{Achsschenkellenkung bei Starrachsen}
    \label{fig:Achsschenkellenkung bei Starrachsen}
\end{figure}
Beide Räder sollen bei Kurvenfahrt auf einer Kreisbahn mit gleichem Mittelpunkt rollen, dem Momentanpol der Bewegung. Bei der Drehschemellenkung ist das automatisch der Fall. Bei einer Einzelradlenkung muss das innere Rad, das näher am Mittelpunkt liegt und dem äußeren vorauseilt, stärker einlenken als das äußere. Bei der Achsschenkellenkung müssen die Räder unterschiedlich geschwenkt werden, das außen liegende weniger stark als das innen liegende. Das erreicht man mit der Bildung des sogenannten Lenktrapezes aus dem Achskörper, den beiden leicht nach innen weisenden Lenkhebeln an den Achsschenkeln und einer Spurstange, die kürzer als der Achskörper ist. Dadurch entstehen beim Schwenken der Räder ungleich lange wirksame Hebelarme, so dass sich die Verlängerungen aller Radachsen ungefähr im Kurvenmittelpunkt schneiden (Ackermann-Prinzip). Ein Hinweis auf ein richtig dimensioniertes Lenktrapez ist die Tatsache, dass sich die verlängerten Lenkhebel an den Achsschenkeln in Geradeausstellung in der Mitte der Hinterachse treffen (siehe oben in obiger Abbildung).\\

Wenn nach rechts gesteuert wird werden diese für den Winkel der beiden Räder verwendetet:
\begin{align}
    \textrm{Linkes Rad}& =\arctan^{-1}\left(\frac{\textrm{Achsenabstand}}{\textrm{KurvenRadius}+\frac{\textrm{RadAbstand}}{2}}\right)\\
    \textrm{Rechtes Rad}& =\arctan^{-1}\left(\frac{\textrm{Achsenabstand}}{\textrm{KurvenRadius}-\frac{\textrm{RadAbstand}}{2}}\right) 
\end{align}

%Wie wurde die Formel im Code umgesetzt?
Im ScrWheelcontroller Script wird der Ackermannwinkel berechnet wird. Der berechnete Ackermannwinkel wird an die vordernen beiden Räder weitergegeben. Die Räder werden zuvor in einer Liste hinzugefügt.
%Warum wurde es so umgesetzt?
Es brauchte ein Controller Script verwendet um die Struktur übersichtlicher zugestalten. Im Wheelscripts werden public bools angelegt die im Inpector dort wird fest gelegt um welches Rad es sich handelt.


\begin{lstlisting}
private void manuelSteering()
{
            
    if (steerInput > 0)
    {
        ackermannAngleLeft = Mathf.Rad2Deg * Mathf.Atan(wheelBase / (turnRadius + (rearTrack / 2))) * steerInput;
        ackermannAngleRight = Mathf.Rad2Deg * Mathf.Atan(wheelBase / (turnRadius - (rearTrack / 2))) * steerInput;
    }
    else if (steerInput < 0)
    {
        ackermannAngleLeft = Mathf.Rad2Deg * Mathf.Atan(wheelBase / (turnRadius - (rearTrack / 2))) * steerInput;
        ackermannAngleRight = Mathf.Rad2Deg * Mathf.Atan(wheelBase / (turnRadius + (rearTrack / 2))) * steerInput;
    }
    else
    {
        ackermannAngleLeft = 0;
        ackermannAngleRight = 0;
    }
}    
\end{lstlisting}


\subsection{Kontrolle der Geschwindigkeit}
%Einleitung
Die Kontrolle der Geschwindigkeit in Unity-Fahrzeugsimulationen ist von entscheidender Bedeutung, um realistisches Fahrverhalten zu gewährleisten.\\

%Funktionsweise der Beschleunigung
In Unity werden die Fahrzeugbewegungen oft durch die Verwendung von Unity Wheel Colliders simuliert. Die Beschleunigung eines Fahrzeugs in Unity erfolgt im Wesentlichen durch die Anwendung eines Drehmoments auf die Unity Wheel Colliders.\\

Um das Fahrzeug zu beschleunigen, wird ein Motor-Drehmoment auf die Unity Wheel Colliders angewendet. Dieses Drehmoment erzeugt eine Rotationskraft an den Rädern, die das Fahrzeug vorwärts bewegt.\\

Das Motor-Drehmoment wird von den Unity Wheel Colliders aufgenommen und in eine Drehbewegung der Räder umgesetzt. Die Räder wiederum übertragen diese Drehbewegung auf den Boden, wodurch das Fahrzeug in Bewegung gesetzt wird.\\

Durch die kontinuierliche Anwendung des Motor-Drehmoments erhöht sich die Geschwindigkeit des Fahrzeugs im Laufe der Zeit, solange keine Gegenkräfte wie Reibung oder Luftwiderstand wirken.\\

Das Bremsen in Unity-Fahrzeugsimulationen wird ebenfalls über die Unity Wheel Colliders simuliert. Es erfolgt durch das Anwenden eines Bremsmoments auf die Räder oder Bremskomponenten.\\

Beim Bremsen wird ein Bremsmoment auf die Unity Wheel Colliders ausgeübt. Dieses Moment erzeugt eine Bremskraft, die das Fahrzeug verlangsamt.\\

Durch die kontinuierliche Anwendung des Bremsmoments nimmt die Geschwindigkeit des Fahrzeugs ab, bis es schließlich zum Stillstand kommt oder bis der Bremsvorgang beendet wird.\\

Die Geschwindigkeitskontrolle in Unity-Fahrzeugsimulationen wird durch die gezielte Steuerung von Beschleunigung und Bremsen sowie durch die Anpassung von Parameterwerten erreicht.\\

Um die Geschwindigkeit zu begrenzen, kann eine maximale Geschwindigkeitsgrenze festgelegt werden. Wenn diese Grenze erreicht ist, wird das Motor-Drehmoment reduziert oder das Bremsmoment erhöht.\\


\begin{figure}[htp]
    \centering
    \includegraphics[width=15cm]{images/Geschwingkeit.pdf}
    \caption{Geschwindigkeits Graph}
    \label{fig: Geschwindigkeit}
\end{figure}

Die Geschwindigkeitskontrolle in Unity-Fahrzeugsimulationen erfordert die präzise Steuerung von Beschleunigung und Bremsen durch die Anwendung von Drehmomenten auf Unity Wheel Colliders. Die Beschleunigung erfolgt durch das Anwenden von Motor-Drehmomenten, während das Bremsen durch das Anwenden von Bremsmomenten simuliert wird. Eine genaue Geschwindigkeitskontrolle ermöglicht eine realistische Simulation von Fahrzeugbewegungen und ist entscheidend für eine authentische Spielerfahrung. Die Anpassung von Motorleistung, Bremsen und physikalischen Parametern spielt eine wichtige Rolle bei der Feinabstimmung der Geschwindigkeitskontrolle in Unity.\\
%wie funktioniert die Beschleunigung?

%Wie funktioniert die Bremse?

%Wie kontrolliert man die Geschwindigkeit?

%Graph einbinden


%Diese muss im späteren verlauf verändert werden vorübergehen geht es in ordnung
%Die beschreibung am besten so durch führen wie es in dem einem Youtube video gezeigt wurde
\subsection{Steuerung von Fahrzeugbewegungen durch Fahranweisungen}

\begin{figure}[htp]
    \centering
    \includegraphics[width=(\textwidth)]{images/AGV_bewegt_sich.PNG}
    \caption{AGV Bewegung}
    \label{AGV Bewegung}
\end{figure}
Die Steuerung von Fahrzeugbewegungen in Simulationen ist von entscheidender Bedeutung für die Schaffung von realistischen Szenarien.\\

Fahranweisungen sind Anweisungen, die einem Fahrzeug in einer Computersimulation gegeben werden, um seine Bewegungen zu steuern. Diese Anweisungen können verschiedene Aspekte des Fahrverhaltens beeinflussen, einschließlich Beschleunigung, Lenkung, Bremsen und mehr. \\

Fahranweisungen beginnen mit der Erfassung von Benutzereingaben oder anderen Steuersignalen. Dies können Controller-Eingaben sein.\\

Die umgewandelten Anweisungen werden auf das Fahrzeug angewendet, indem die erforderlichen physikalischen Parameter, wie Motor-Drehmoment, Lenkwinkel oder Bremsmoment, entsprechend angepasst werden.\\

Fahranweisungen können über einen bestimmten Zeitraum hinweg angewendet werden, um eine präzise Steuerung der Bewegungen sicherzustellen. Timer oder Zeitbegrenzungen können verwendet werden, um die Dauer von Aktionen zu bestimmen.\\
%Wie wird die Spurr gezeichnet?

%Wie wird gesteuert?
