
\chapter{Notizen}
\subsection{Aufgaben}
Die Aufgabe ist es sich in Unity ein zuarbeiten. Dort eine Scenerie aufzubauen und die Daten die die vier Sensoren liefern mit den Bewegungs vektoren übertragen.
Die übertragung soll über einen gRPC Client und einen gRPC Server laufen. Dannach sollen mit den erzeugten Daten das Feld des DSM Host in Unity angezeigt werden(Best Case für Demo gedacht).
\begin{itemize}
    \item Zeitplan
    \item Riskomanagment
    \item 
\end{itemize}
\subsection{Zeitplan}

\begin{itemize}
    \item 13. Juli - 21. Juli = 7 Tage
    \item 4. September - 30. September = 20 Tage
    \item 27 Arbeitstage a 7 Stunden gleich 189 Stunden
\end{itemize}



% Please add the following required packages to your document preamble:
% \usepackage[table,xcdraw]{xcolor}
% If you use beamer only pass "xcolor=table" option, i.e. \documentclass[xcolor=table]{beamer}

\begin{landscape}
    
\begin{tabularx}{21cm}{|l|l|l|X|X|X|X|X|X|X|X|X|X|X|X|}
    \hline
    \textbf{Aufgabe}                       & \textbf{Anfang} & \textbf{Ende} & \textbf{28}            & \textbf{29}            & \textbf{30}            & \textbf{31}            & \textbf{32}            & \textbf{33}            & \textbf{34}            & \textbf{35}            & \textbf{36}            & \textbf{37}            & \textbf{38}            & \textbf{39}            \\ \hline
    Planung+ Aufgabe Prio 1                & 12.07           & 14.07         & \cellcolor[HTML]{34CDF9} &                          &                          &                          &                          &                          &                          &                          &                          &                          &                          &                          \\ \hline
    Aufgabe Prio 2                         & 17.07           & 21.07         &                          & \cellcolor[HTML]{34CDF9} &                          &                          &                          &                          &                          &                          &                          &                          &                          &                          \\ \hline
    Abwesenheit                            & 24.07           & 01.09         &                          &                          & \cellcolor[HTML]{C0C0C0} & \cellcolor[HTML]{C0C0C0} & \cellcolor[HTML]{C0C0C0} & \cellcolor[HTML]{C0C0C0} & \cellcolor[HTML]{C0C0C0} & \cellcolor[HTML]{C0C0C0} &                          &                          &                          &                          \\ \hline
    Aufgabe Prio 2                         & 04.09           & 08.09         &                          &                          &                          &                          &                          &                          &                          &                          & \cellcolor[HTML]{34CDF9} &                          &                          &                          \\ \hline
    Aufgabe Prio 3                         & 11.09           & 15.09         &                          &                          &                          &                          &                          &                          &                          &                          &                          & \cellcolor[HTML]{34CDF9} &                          &                          \\ \hline
    Aufgabe Prio 4+Doku                    & 18.09           & 29.09         &                          &                          &                          &                          &                          &                          &                          &                          &                          & \cellcolor[HTML]{68CBD0} & \cellcolor[HTML]{68CBD0} & \cellcolor[HTML]{68CBD0} \\ \hline
    Praxisbericht + Kollogium Vorbereitung & 18.09           & 22.09         &                          &                          &                          &                          &                          &                          &                          &                          &                          &                          & \cellcolor[HTML]{34FF34} &                          \\ \hline
    Kollogium                              & 25.09           & 29.09         &                          &                          &                          &                          &                          &                          &                          &                          &                          &                          &                          & \cellcolor[HTML]{FE0000} \\ \hline
\end{tabularx}
\end{landscape}


Prios der arbeit
Prio 1 Senordaten über gRPC übertragen
Prio 2 Wenderadius und Geschwindigkeit übertragen mit der Thales (Einauto selbst Desginen)
es gibt mehrer auto: gabelstapler, Panzerrolle, normale Auto
Prio 3 Polghonzug erhalten und in Unity darstellen
Prio 4 NotSignal erhalten und das AGV anhalten 

\subsection{Notizen 03.07}
Unity zum laufen bekommen auf dem anderen PC.

\subsection{12.07}
Plan für heute dem AGV Geschwindigkeit geben. Und weiter geben. Das Problem die AGV/Service wird nicht erkannt. Obwohl der SplineWalker mit dem dem Agv Service Script verbunden ist. Die 

\subsection{13.07}
\begin{itemize}
\item Projekt Einsicht in VS Studio
\item Martin schreiben wegen dem AGV übertragung
\item Fragen formulieren zur Thales Schnittstelle
\item 
\end{itemize}

Fragen zur Thales Schnittstelle
\begin{itemize}
\item Jan hat eine Agvservice script erstellt welches IAgv verwendet aber es wird nicht bei den Avalibele Services angezeigt
\item Hallo Martin wir haben noch ein paar Fragen zur Thales Api. Hast du heute Nachmittag Zeit für ein Meeting mit Artem und mir?
\end{itemize}

\subsection{14.07}
\begin{itemize}
\item Sprint Meeting vorbereiten
\item die vier Scanner abfragen und übertragen. Wie funktioiert das mit dem Stepper?
\item Welche Scanner sollen übertragen werden?
\item Wie werden AGVs in Unity umgesetzt?
\item Aufgaben in Jira 
\end{itemize}

\subsection{17.07}
\begin{itemize}
\item Doku erledigt
\item Nächste Aufgabe erstellen es AGV
\end{itemize}

Nächste Aufgabe erstellen es AGV
\begin{itemize}
\item Scene aufbauen
\item Object für des AGV erstellen 
\end{itemize}

\subsection{18.07}
Aufgaben für heute 
\begin{itemize}
\item Zeitplan überarbeiten welche Aufgaben wurden genau bearbeitet orientieren an den Jira Tickets
\item Risikoanalyse stellen
\item wie könnte ein Polygonzug übergeben werden? es können nur Parameter geschickt werden die in den Protobuff passen
\item Inhaltsverzeichnis für Bericht erstellen
\item Git zum laufen bringen 
\item Arbeitsparkte für AGVs erstellen
\end{itemize}

Projektplan



Anforderungen von Manfred
\begin{itemize}
\item AutoStepper verstehen muss alle 10 ms senden
\item Abstand zwischen Rad und lenkachse muss freibestimmbar sein
\item Es muss ein Dreirädrieges Fahrzeug sein
\item es muss mit angaben eins Scripts laufen 
\item Aufgaben pakete müssen in Jira stehen
\end{itemize}

\subsection{19.07}
\begin{itemize}
\item gRPC Schnittstelle bearbeiten
\item Inhaltsverzeichnis erstellen
\end{itemize}

Inhaltsverzeichnis
\begin{itemize}
\item Einleitung
\begin{itemize}
\item Projektumgebung
\end{itemize}
\item Aufgabenstellung
\begin{itemize}
\item Interesse der Firma
\item Generelle Aufgabe
\end{itemize}
\item Grundlagen
\begin{itemize}
\item gRPC 
\item Thales
\item Senoren in der Viruellen Umgebung
\end{itemize}
\item Konzept
\begin{itemize}
\item Design der Sensoren in der Virtuellen Umgebungen 
\item AGVs in der Virtuellen Umgebung
\end{itemize}
\item Implemenierung
\begin{itemize}
\item Implementierung der Schnittstelle
\item Echtzeitübertragung der Sensoren
\end{itemize}
\item Ergebnis
\item Ausblick
\end{itemize}


Fragen an Manfred
\begin{itemize}
\item Welche Scanner genau
\item 
\end{itemize}


\subsection{20.07}

Aufgaben für heute
\begin{itemize}
\item Sensoren Desgin --> feritg
\item GitPush --> fertig
\item Dokumentiern was gemacht wurde --> fertig
\item namen für Directory heraus suchen --> feritg
\item Urlaub buchen --> fertig
\item Abweseheit Email eintragen --> fertig


\end{itemize}

\subsection{21.08}
\begin{itemize}
\item Einen richtigen Zeitplan erstellen am besten mit Jira
\item Verstehen was gemacht wurde
\item was sind nochmal meine Aufgaben
\item Zeitplan für den t2000 erstellen 
\item Kollogium vorbereiten 
\item Aktuelle wichtige aufgabe Version umstellen
\item darstellung für den Zeitplan finden
\item Struktur für den t2000 festlegen
\item Wendekreis berechnen
\end{itemize}

\subsection{22.08}
\begin{itemize}
\item \url{https://www.youtube.com/watchv=x0LUiE0dxP0&list=PL6sL8yhcAsqVEWS5lzeWu3yA1LIb9rxx5&index=1}
\item Fahrzeug erstellen
\item 
\end{itemize}


\subsection{23.08}
\begin{itemize}
\item Plan machen um das Fahrzeug zu erstellen
\item es soll Ackermann Steering genutzt werden 
\item und der Code der Suspension soll sehr sauber aussehen damit man es besser erklären kann
\item 
\end{itemize}


\subsection{24.08}
\begin{itemize}
\item Es sollten doch 
\item Fahrzeug dynamik anpassen nach einem echten AGV
\item Gabelstapler anfangen 
\item was soll als nächsten geschrieben werden?
\item struktur für die Methodik und Grundlagen festlegen
\item 
\end{itemize}

\subsection{04.09}
\begin{itemize}
\item line malen hinter dem AGV
\item steuerungs script erstellen
\item
\end{itemize}


\subsection{17.09}
Schritte um die Räder umzustellen

Eigenschaften des Rads 
Damping sollte unverändert bleiben
Suspention Distance sollte 0 sein in meinem Fall
Spring und Damper sollte beim Default bleiben
Der Rest sollte gleich bleiben

Tipps für die Fahranweisung es könnte auch mit Deltatime umgesetzt werden 
private void Update()
    {
        // Überprüfen, ob das Gaspedal gedrückt werden soll
        if (gasTimer < gasDuration)
        {
            ApplyGas();
            gasTimer += Time.deltaTime;
        }

        // Überprüfen, ob gelenkt werden soll
        if (steerTimer < steerDuration)
        {
            ApplySteer();
            steerTimer += Time.deltaTime;
        }
    }

Zum Beispiel so







