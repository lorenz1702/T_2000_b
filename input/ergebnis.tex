\chapter{Ergebnis}

In diesem wissenschaftlichen Bericht wurden die Grundlagen und die Implementierung eines Systems zur Steuerung von Fahrzeugbewegungen in Unity durch Fahranweisungen behandelt. Dieses System ermöglicht die Verbindung mit der ThalesAPI zur Weitergabe von Scannerdaten und Variablen, was eine wichtige Grundlage für die Entwicklung von Algorithmen zur Personenerkennung und die Simulation verschiedener Szenarien darstellt.\\


Die Implementierung der Schnittstelle zwischen Unity und der ThalesAPI wurde erfolgreich durchgeführt. Dies ermöglichte die Übertragung von Scannerdaten und Variablen in Echtzeit, was zuvor in dieser Form nicht möglich war. Die Integration von Sensorinformationen in virtuelle Umgebungen eröffnet neue Möglichkeiten für die Entwicklung und das Testen von autonom gesteuerten Fahrzeugen.\\

Die Umsetzung der Fahrphysik in Unity war eine Herausforderung, insbesondere die Lenkung und die Kontrolle der Geschwindigkeit. Dennoch konnte ein vier-rädriges AGV entwickelt werden, das als Plattform für weitere Tests und Experimente dient.\\

\section{Ausblick}

Die Entwicklung der Fahrphysik gestaltete sich aufwendiger als erwartet, und es konnte bisher nur ein vier-rädriges AGV entwickelt werden. Die Integration von echten Signalen des Sichtungsalgorithmus steht noch aus, da die Entwicklung zu diesem Zeitpunkt noch nicht abgeschlossen war. Jedoch können Signal von außen erhalten werden. Ebenso konnte die Umsetzung von Polygonzügen bisher nicht realisiert werden.\\

Für zukünftige Forschung und Entwicklung bietet dieses System jedoch viel Potenzial. Es ermöglicht nicht nur die Generierung von Testdaten, sondern auch die Schaffung und Simulation verschiedener Szenarien. Die Weiterentwicklung der Fahrphysik und die Integration fortgeschrittener Sensortechnologien werden die Grundlage für weiterführende Arbeiten in diesem Bereich bilden.\\

Insgesamt zeigt diese Arbeit, dass die Verbindung von Unity, Fahranweisungen und Sensorinformationen ein vielversprechender Ansatz für die Entwicklung und Evaluierung autonomer Fahrzeuge und Algorithmen zur Personenerkennung ist. Es bleibt spannend, wie sich dieser Bereich in Zukunft weiterentwickeln wird.\\


%Präsentiere deine Forschungsergebnisse in klar strukturierter Form. Verwende Tabellen, Grafiken oder Diagramme, um deine Ergebnisse zu veranschaulichen. Interpretiere die Ergebnisse und beantworte deine Forschungsfragen.


%Interpretiere deine Ergebnisse im Kontext des theoretischen Hintergrunds. Diskutiere, wie deine Ergebnisse zu den bestehenden Erkenntnissen passen oder davon abweichen. Zeige auf, welche Implikationen deine Arbeit hat und welche offenen Fragen bleiben.